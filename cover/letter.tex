% Cover letter using letter.sty
\documentclass{letter} % Uses 10pt
%Use \documentstyle[newcent]{letter} for New Century Schoolbook postscript font
% the following commands control the margins:
\topmargin=-1in    % Make letterhead start about 1 inch from top of page 
\textheight=8in  % text height can be bigger for a longer letter
\oddsidemargin=0pt % leftmargin is 1 inch
\textwidth=6.5in   % textwidth of 6.5in leaves 1 inch for right margin

\begin{document}

\signature{Zhechen Min}           % name for signature 
\longindentation=0pt                       % needed to get closing flush left
\let\raggedleft\raggedright                % needed to get date flush left
 
 
\begin{letter}{}


\begin{flushleft}
{\large\bf Zhechen Min}
\end{flushleft}
\medskip\hrule height 1pt
\begin{flushright}
\hfill 7541 Colleen Street, Burnaby, B.C. Canada V5A 2A5 \\
\hfill +1-778-862-3661, zmin@sfu.ca
\end{flushright} 
\vfill % forces letterhead to top of page

 
\opening{Dear IBM,} 
 
\noindent IBM is one of the greatest IT company in the world. As a programmer, becoming a member of IBM is always my dream. Now I'm trying to realize my dream by applying the co-op position.

\noindent Programming is my top interest since I was 14. With about 10 years programming experience, plus the zeal for writing code, I still have lots of other abilities that make me a fabulous programmer. 

\noindent I am able to learn and use new knowledge very fast. When I was in RIM as a co-op developer for MVS system, a telephone system for large company, there are lots of technologies like dtmf-tone and PSTN that I even never heard at the very beginning. In MRX Solutions, I was asked to write code on Mac which I haven't tried at that time. However I can always get the general graph of these new technologies and start my work quickly in one or two weeks. In ACM/ICPC programming competition, once I learnt a new algorithm and implemented it just in one hour. New knowledge or technology never scares me off because I have confidence in getting familiar with it quickly

\noindent I'm an active member of ACM/ICPC team since I entered the university. ACM/ICPC is the top algorithm competition for university students all over the world. This famous competition attracts all the students who love programming and challenge from every countries every year. Using the knowledge of algorithm and data structure, plus with the ability of writing and debugging code in a short time are the basic requirments for contestants. What's more, how to optimize the code based on different situations, and the ability of explaining your idea to you teammates are also the skills I gained from it. 
\noindent However, the most valuable thing I learned from the two co-op terms' experience and ACM/ICPC competition is the importance of team work. Teamwork is not just about gathering people to solve a problem quickly, but a more crucial point is to support and help each other so everyone is fareless. Working with the smart people is very enjoyable and I believe IBM can give me such environment. On the other way around, my ability and skills can meet the requirement of IBM. Thank you for your time and I'm looking forward to hearing from you. 

\closing{Sincerely,} 
 

 m%\encl{}  				% Enclosures

\end{letter}
 

\end{document}






